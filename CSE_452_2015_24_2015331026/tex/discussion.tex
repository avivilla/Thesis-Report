\documentclass{standalone}
\usepackage{standalone}

\begin{document}
\chapter{Discussion}
\label{chap:Discussion}
Here we will have a short analysis of our works and the places of improvement in this work.
\section{Result Analysis}
So far now, we have trialed with 3 different Architecture and rectified 2 major drawbacks. The whole work’s output can be shown in a single table \ref{finaltable}
\begin{table}[h!]
    \centering
        \begin{tabular}{|c|c|}
            \hline
            \textbf{model} & ]\textbf{accuracy} \\ [1ex]
            \hline
            initial model & 51.71\% \\
            \hline
             Bi-LSTM & 55.77\% \\
              \hline
                initial model + Higher Probability First & 72.9\%\\
            \hline
             Sequential model + Higher Probability First & 83.8\%\\
            \hline
           
            initial model + Higher Probability First + Higher Class
            Difference & 84.7\%  \\ 
            \hline
            Bi-LSTM + Higher Probability First & 85.56\%\\
            \hline
            Sequential model + Higher Probability First + Higher Class
            Difference & 89.45\%  \\ 
            \hline
            Bi-LSTM + Higher Probability First + Higher Class
            Difference & 90(89.98)\%  \\ 
            \hline
        \end{tabular}
        \caption{experiment results with initial model}
        \label{finaltable}
\end{table}
From table \ref{finaltable}, it is clear that without using Higher Probability First and Higher Class Difference, the accuracy is less than 60\% in each model. But using Higher Probability First, it jumps to almost 85\%. Adding Higher Class Difference we can achieve the best result of almost 90\%. So we can say that without rectifying these two drawbacks we could not get a satisfactory result.
\section{Scopes of Works}
We can not say that the work is anywhere close to the satisfactory works done on other languages. So there are a lot of things that can be improved.
\subsection{A Corpus of word and tags}
We had to use an XML based dictionary for handling unknown words problems in a very naive way discussed in section \ref{}. So we need a huge data corpus to provide us with more words and tags. That will strengthen the base of this works, feeding probability as a feature.

\subsection{CRF based Model}
The best result in Bengali POS Tagging has been found using the CRF based model. We could not work with that. So we are hopeful of a good result using that technique.
\subsection{Merging with traditional features}
Using probability as a feature along with the traditional morphological features can be a huge advantage in the work of this field. This can enhance the chance to get a better model.









\end{document}